% For general terms, set 'first=' to control what prints on first use.
\newglossaryentry{embedding}{
  name={embedding},
  description={a numeric vector representation of text in a fixed-dimensional space},
  first={an embedding (a numeric vector representation of text in a fixed-dimensional space)},
  plural={embeddings},
  firstplural={embeddings (numeric vector representations of text in a fixed-dimensional space)}
}

\newglossaryentry{vectorsearch}{
  name={vector search},
  description={retrieval based on distances in an embedding space},
  first={vector search (retrieval based on distances in an embedding space)}
}

\newglossaryentry{qrel}{
  name={qrel},
  plural={qrels},
  description={short for query relevance judgment; a relevance judgment mapping a query to a document with an assigned relevance label},
  first={qrel (query relevance judgment)},
  firstplural={qrels (query relevance judgments)}
}

\newglossaryentry{zeroshot}{
  name={zero-shot prompting},
  description={a prompting technique where a language model is given only the task description without any examples and is expected to perform the task directly based on its prior knowledge},
  first={zero-shot prompting (providing only the task description without examples)},
}

\newglossaryentry{oneshot}{
  name={one-shot prompting},
  description={a prompting technique where the model is given exactly one example of the task along with the task description, helping it generalize the pattern to new inputs},
  first={one-shot prompting (providing exactly one example along with the task description)},
}

\newglossaryentry{fewshot}{
  name={few-shot prompting},
  description={a prompting technique where the model is given a small number of input–output examples along with the task description to guide its behavior on new inputs},
  first={few-shot prompting (providing a few input–output examples along with the task description)},
}

